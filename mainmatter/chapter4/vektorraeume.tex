\section{Vektorräume}
%
%
%
\subsection{Vektorraum}
%
\subsubsection{Definition :}
Ein $K$-Vektorraum $V$ ist eine Menge für die gilt:
\begin{itemize}
	\item $\vec{0}\in V$
	\item für je zwei Vektoren $\vec{v},\vec{w}\in V$ ist $\vec{v}+\vec{w}$ erklärt und $v \in V, 
		\, \lambda \in \mathbb{R} \, \lambda \cdot \vec{v} \in V$
\end{itemize}
%
\begin{multicols}{2} 
\begin{enumerate}
	\item $\vec{v} + \vec{w} = \vec{w} + \vec{v}$
	\item $(\vec{v}+\vec{w})+\vec{u}=\vec{v}+(\vec{w}+\vec{u})$
	\item $\vec{v}+0=\vec{v}$
	\item für jedes $\vec{v} \in V$ gibt es $x \in V:\vec{v}+x=0$
	\item $1\cdot \vec{v}=\vec{v}$
	\item $\lambda\cdot(\vec{u}+\vec{v})=\lambda\vec{u}+\lambda\vec{v}$
	\item$(\lambda+\mu)\vec{v}=\lambda\vec{v}+\mu\vec{v}$
	\item $(\lambda \cdot \mu)\cdot \vec{v} = \lambda \cdot (\mu\cdot\vec{v})$
\end{enumerate}
\end{multicols}
%
%
%
\subsubsection{Bemerkung :}
\begin{enumerate}
	\item $K$-Vektorraum abgekürzt mit $VR$
	\item negativer Vektor $(-\vec{v})$\\
		Der negative Vektor ist eindeutig: $\vec{v}+\vec{w}=0 \Rightarrow 
		\vec{w}=\vec{w}+(\vec{v}+(-\vec{v}))=0+(-\vec{v})=-\vec{v}$
\end{enumerate}
%
%
%
\subsubsection{Beispiel :}
\begin{enumerate}
	\item $K^{n}=\{\begin{pmatrix} a_{1} \\ \vdots \\ a_{n} \end{pmatrix} : a \in K\}$ mit $+ 
		\begin{pmatrix} a_{1} \\ \vdots \\ a_{n} \end{pmatrix} + \begin{pmatrix} b_{1} \\ 
		\vdots \\ b_{n } \end{pmatrix} = \begin{pmatrix} a_{1} + b_{1} \\ \vdots \\ 
		a_{n}+b_{n} \end{pmatrix}$\\
		$\lambda \in K: \lambda\begin{pmatrix}a_{1} \\ \vdots \\ a_{n}\end{pmatrix} =
		\begin{pmatrix} \lambda a_{1} \\ \vdots \\ \lambda a_{n} \end{pmatrix}$
	\item $(m\times n)$ - Matrizen $K^{m\times n}$\\
		$\begin{pmatrix} a_{11} & \cdots & a_{1n}\\
						\vdots & & \vdots\\
						a_{m1} & \cdots & a_{mn} \end{pmatrix} + 
		\begin{pmatrix} b_{11} & \cdots & b_{1n}\\
						\vdots & & \vdots\\
						b_{m1} & \cdots & b_{mn} \end{pmatrix} = 
		\begin{pmatrix} b_{11}+a_{11} & \cdots & b_{1n}+a_{1n}\\
						\vdots & & \vdots\\
						b_{m1}+a_{m1} & \cdots & b_{mn}+a_{mn} \end{pmatrix}$\\
		$\lambda \cdot 
		\begin{pmatrix} a_{11} & \cdots & a_{1n}\\
						\vdots & & \vdots\\
						a_{m1} & \cdots & a_{mn} \end{pmatrix} = 
		\begin{pmatrix}\lambda a_{11} & \cdots &\lambda a_{1n}\\
						\vdots & & \vdots\\
						\lambda a_{m1} & \cdots &\lambda a_{mn} \end{pmatrix}$
\end{enumerate}
%
%
%
\subsubsection{Bemerkung :}
Für eine Matrix gibt es abkürzende Schreibweise: $(a_{ij})$, i-te Zeile, j-te Spalte steht $a_{ij}$. 
%
%
%
\subsection{Untervektorraum}
%
\subsubsection{Definition :}
$V: K-VR$. $U\subset V$ heißt $K$-Unterraum, wenn $0 \in U$ und aus $a,b \in U: a+b\in U, \lambda \cdot a \in U \, \forall \lambda \in K$\\
Schreibweise: $U \leq V$ wobei $\leq$ hier bedeutet, das $U$ in $V$ enthalten ist und strukturerhaltend ist. 
%
%
%
\subsubsection{Beispiel :}
\begin{itemize}
	\item $U=\{0\}\, U=V$
	\item in $\mathbb{R}^{2}:$ Geraden durch den Ursprung
	\item in $\mathbb{R}^{3}:$ Ebene durch den Nullpunkt alle Unterräume vom Typ 	
		$\{0\}$, Gerade, Ebene, $\mathbb{R}^{3}$
	\item $\mathbb{C}^{2}: \lambda \cdot \begin{pmatrix} z_{1} \\ z_{2} \end{pmatrix}$
\end{itemize}
%
%
%