\section{Aussagen}
\begin{center}
\begin{tabular}{l|c|c}
Aussage & w & f\\ \hline
Wasser ist nass & x & \\
A. Merkel ist Bundeskanzlerin & x & \\
Rößler wäre gern Bundeskanzler & ? & ?\\
Ein Kaninchen ist eine Pflanze & & x\\
Ein Dreieck hat vier Ecken & & x\\
Jede gerade Zahl größer 2 ist Summe zweier Primzahlen - Goldbach Vermutung & ? & ? \\
Wenn 2012 Frauenüberschuss bei Matheprofessorinnen herrscht, dann ist die Erde eine Scheibe & x &\\
\end{tabular}

\quad\\
\quad\\

Für "`$ A \Rightarrow B$ ist wahr."' sagt man auch \\
$A$ ist hinreichend für $B$. \\
$B$ ist notwendig für $A$. \\

\quad\\

\begin{tabular}{ll}
	$\neg A=$ nicht $A$ & $A \vee B =$ $A$ oder $B$\\
	$A \wedge B =$ $A$ und $B$ & $A \Leftrightarrow B =$ $A$ ist äquivalent zu $B$\\
\end{tabular}

\quad\\
\quad\\

\begin{tabular}{cc||ccccc}
$A$ & $B$ & $\neg A$ & $A \wedge B$ & $A \vee B$ & $A \Rightarrow B$ & $A \Leftrightarrow B$\\ \hline
w & w & f & w & w & w & w \\
w & f & f & f &w & f & f \\
f & w & w & f & w & w & f \\
f & f & w & f & f & w & w\\
\end{tabular}
\end{center}
%
%
%
\subsection{Satz:}
$A,B,C$ seien Aussagen. Folgende Aussagen sind wahr: "`Tautologie"' (Der Beweis wird durch die Wahrheitstafel erbracht)
\begin{center}
\begin{tabular}{cc||ccc}
$A$ & $\neg A$ & $A \vee \neg A$ & $\neg(A\vee \neg A)$ & $\neg(\neg A)$\\ \hline
w & f & w & w & w\\
f & w & w & w & f\\
\end{tabular}
\end{center}
\quad\\
\begin{enumerate}
\item $A\vee(\neg A)$
\item $\neg (A \wedge \neg A)$
\item $\neg (\neg A) \Leftrightarrow A$
\item $\neg (A \wedge B) \Leftrightarrow \neg A \vee \neg B$ \ z.B. $A=$ Die Sonne scheint \ $B=$ Es ist bewölkt
\item $\neg (A \vee B) \Leftrightarrow \neg A \wedge \neg B$ \ z.B. $A=$ Wasser ist trocken \ $B=$ Es ist Sommer
\item $(A\Rightarrow B) \Leftrightarrow (\neg B \Rightarrow \neg A)$ \ $A=$ Es blitzt \ $B=$ es donnert
\item $A \wedge (A \Rightarrow B) \Rightarrow B$
\item $A \Rightarrow B \wedge \neg B \Rightarrow \neg A$
\item $[(A\Rightarrow B) \wedge(B\Rightarrow C)]\Rightarrow(A\Rightarrow C)$
\item $A\wedge (B\vee C)\Rightarrow (A\wedge B)\vee(A\wedge C)$
\item $A \vee(B\wedge C) \Rightarrow (A\vee B) \wedge (A \vee C)$
\end{enumerate}
4 und 5 sind die De Morgan'sche Gesetze. 7 Ist der Modus ponens, 8 Modus tollens und die 9 Modus barbara (=Transitivität)\\
%
%
%
