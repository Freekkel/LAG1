\section{Geraden und Ebenen}
\subsubsection{Definition}
Gerade : $\vec{b} + \mathbb{R}\vec{a}$, wobei $\vec{a}\neq\vec{0}$\\
 $\vec{a}$ heißt Richtungsvektor, $\vec{b}$ heißt Stützvektor\\
Ebene: $\vec{c}+\mathbb{R}\vec{a}+\mathbb{R}\vec{b}$\\
$\vec{a}, \vec{b}$ sind Richtungsvektoren, $\vec{c}$ ist der Stützvektor\\
%
%
%
$x_{1}=c_{1}+\lambda a_{1}+\mu b_{1}$\\
$x_{2}=c_{2}+\lambda a_{2}+\mu b_{2}$\\
$x_{3}=c_{3}+\lambda a_{3}+\mu b_{3}$\\

$d=<\vec{n},\vec{x}>=<\vec{n},\vec{c}>+\lambda<\vec{n},\vec{a}>+\mu<\vec{n},\vec{b}$\\
$\Rightarrow <\vec{n},\vec{a}>=0=<\vec{n},\vec{b}>$\\
gesucht ist der Normalenvektor
%
%
%
\subsubsection{Definition:}
$\vec{a},\vec{b} \in \mathbb{R}^{3} \qquad \vec{a} \times \vec{b} = (a_{2}b_{3}-a_{3}b_{2},a_{3}b_{1}-a_{1}b_{3}, a_{1}b_{2}-a_{2}b_{1})$
%
%
%
\subsubsection{Rechenregeln:}
$(\vec{a}\times\vec{b})=-(\vec{b}\times\vec{a})$\\
$(\vec{a}+\vec{b})\times\vec{c}=\vec{a}\times\vec{c}+\vec{b}\times\vec{c}$\\
$t(\vec{a}\times\vec{b})=(t\vec{a})\times\vec{b}=\vec{a}+(t\vec{b})$\\
$<\vec{a},\vec{b}\times\vec{c}>=<\vec{b},\vec{c}\times\vec{a}>=<\vec{c},\vec{a}\times\vec{b}>$
%
%
%
\subsubsection{Satz:}