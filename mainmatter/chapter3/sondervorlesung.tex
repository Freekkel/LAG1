\section{Sondervorlesung Komplexe Zahlen}
%
%
%
\subsection{Einführung}
$\mathbb{N}_{0}\subset\mathbb{Z}\subset\mathbb{Q}\subset\mathbb{R}$
%
%
%
\subsubsection{$\mathbb{N}_{0}\rightarrow\mathbb{Z}$:}
$\mathbb{Z}:=\mathbb{N}_{0}\times\mathbb{N}_{0}$ (Betrachte Paare von natürlichen Zahlen)\\
\qquad\\
Äquivalenzklassen:\\
$(n,m)\sim(n',m'):\Leftrightarrow n+m' = n'+m \Leftrightarrow n-m=n'-m'$\\
$(n-m)\cdot(n'-m')=nn'+mm'-nm'-n'm$\\
$(7,4)\sim(3,0)$\\
$[n,m]:=\{(n',m')|(n',m')\sim(n,m)\}$\\
\qquad\\
Definiere, was auf diese Menge "`+"' und "`$\cdot$"' sein soll:\\
$[n,m]+[n',m']:=[n'+m,n+m']$\\
$[n,m]\cdot[n',m']:=[nn'+mm',nm'+n'm]$\\
$n:=[n,0]$\qquad $-n:=[0,n]$
%
%
%
\subsubsection{$\mathbb{Z}\rightarrow\mathbb{Q}$:}
$\mathbb{Q}=\mathbb{Z}\times\mathbb{Z}\diagdown N$\\
$(a,b)\sim(a',b'):\Leftrightarrow ab'=ba' \qquad \frac{a}{b}=\frac{a'}{b'}\Leftrightarrow ab'=a'b$\\
$\frac{a}{b}:=[a,b]$\\
Addition: $[a,b]+[c,d]:=[ad+bc,bd]$\\
Multiplikation: $[a,b]+[c,d]:=[ac,bd]$\\
$\Rightarrow$ Multiplikation in $\mathbb{Q}$ ist genauso definiert wie Addition in $\mathbb{Z}$!
%
%
%
\subsubsection{$\mathbb{Q}\rightarrow\mathbb{R}$:}
$\mathbb{R}:=\dotsc$\\
Konstruktion sehr kompliziert, über Zahlenfolgen in $\mathbb{Q}$.
$x^{2}+1=0$ hat in $\mathbb{R}$ keine Lösung\\
$x^{2}\geq0 \, \forall x\in \mathbb{R}\Rightarrow x^{2}+1\geq1>0$\\
Cauchi-Folge: Abstand zwischen Folgeglieder wird immer kleiner trotzdem u.U. kein Grenzwert in $\mathbb{Q}$
%
%
%
\subsubsection{$\mathbb{R}\rightarrow\mathbb{C}$:}
$x^{2}\mp 0 \, \forall x\in\mathbb{R}\Rightarrow x^{2}+\geq>0$\\
Wir suchen einen Körper $\mathbb{C}$ mit 
\begin{enumerate}
	\item $\mathbb{R}\subset\mathbb{C}$
	\item Die Gleichung hat (mindestens) eine Lösung.
\end{enumerate}
Als Menge ist $\mathbb{C}:=\mathbb{R}\times\mathbb{R}=\{(x,y)|x,y\in\mathbb{R}\}$\\
Wir definiere eine Addition durch folgende Vorschrift:\\
$(a,b)+(c,d):=(a+c,b+d)$\\
und eine Multiplikation durch:\\
$(a,b)(c,d):=(ac-bd,ad+bc)$
%
%
%
\subsubsection{Behauptung:}
$(\mathbb{C},+,\cdot)$ ist eine Körper mit $0=(0,0)$ (Nullelement)\\
$1\mathop{=}\limits^{\text{*}}(1,0)$\\
\qquad\\
*($(a,b)(x,y)=(a,b)$\\
$=(ax-by,ay+bx)$\\
$ax-by=a$\\
$bx+ay=b \leadsto(x,y)=(1,0)$)\\
\qquad\\
Betrachtet man Elemente der Form $(x,0) \in \mathbb{C}$, dann gilt:\\
$(a,0)+(b,0)=(a+b,0)$\\
$(a,0)\cdot(c,0)=(ac,0)$\\
\qquad\\
Wir können also $\mathbb{R}$ als Unterkörper von $\mathbb{C}$ (bzw. $\mathbb{C}$ als Körpererweiterung von $\mathbb{R}$) auffassen durch die Identifizierung:\\
$\mathbb{R}\rightarrow\mathbb{C}$\\
$a\rightarrow(a,0)$\\
Was ist Nullstelle von $x^{2}+1=0$?
 \begin{eqnarray*}
	x&=&(a,b)\\
    (0,0)=0=x^{2}+1 &=& (a,b)^{2}+(1,0) \\
	&=& (a,b)(a,b)+(1,0) \leadsto \qquad \text{i.) }0\mathop{=}\limits^{\text{!}}a^{2}-b^{2}+1\\
	&=& (a^{2}-b^{2},2ab)+(1,0) \leadsto \quad \text{ii.) } 0 \mathop{=}\limits^{\text{!}} 2ab\\
	&=& (a^{2}-b^{2}+1,2ab) \leadsto \qquad \quad \text{iii.) } a=0 \text{ oder } b=0\\
  \end{eqnarray*}
\begin{itemize}
	\item $b=0 \Rightarrow 0=a^{2}+1 \quad \lightning$
	\item $a=0 \Rightarrow 0=-b^{2}+1\Rightarrow b = \pm\sqrt{1} = 1$
\end{itemize}
$
x=\left\{
\begin{array}{l}     
    (0,1)\\
    (0,-1)
\end{array}\right.
$\\
\qquad\\
Wir schreiben ab jetzt:\\
$i:=(0,1)$\\
$-i:=(0,-1)$\\
und wir schreiben: $a:=(a,0)$ für $a\in\mathbb{R}$\\
dann lässt sich jede komplexe Zahl wie folgt schreiben:\\
$z=(a,b)\in\mathbb{C}=\underline{a+bi}$\\
.[ 
\begin{eqnarray*}
a+bi &=& (a,0)+(b,0)\cdot(0,1)\\
&=& (a,0)+(0,b\cdot1)\\
&=& (a,b)\\
\end{eqnarray*}
Eine komplexe Zahl ist die Information, zweier reellen Zahlen $a$ und $b$ zu haben und ein Element $i$, das mit sich selbst multipliziert $-1$ ergibt.]\\
$\mathbb{C}$ ist ein 2-dimensionaler $\mathbb{R}$-Vektorraum mit der Basis $(1,i)$.