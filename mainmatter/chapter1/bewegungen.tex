\section{Bewegungen}
%
%
%
\subsubsection{Definition:}
Eine Abbildung heißt Bewegung oder Isometrie wenn $\forall a,b \in \mathbb{R}^{2}\qquad \Vert A(c) - A(b) \Vert = \Vert \vec{a}-\vec{b}\Vert$.
%
%
%
\subsubsection{Satz:}
$A: \mathbb{R}^{2} \rightarrow \mathbb{R}^{2}$ Bewegung mit $A\begin{pmatrix} 0 \\ 0 \end{pmatrix} = \begin{pmatrix} 0 \\ 0 \end{pmatrix}$. Dann ist $A$ linear. Es gibt $c,s \in \mathbb{R}$ mit $c^{2}+s^{2}=1$, sodass die Matrix von $A \, R_{c,s} = \begin{pmatrix} c & -s \\ s & c \end{pmatrix} \, det(R_{c,s}) = 1$ oder $S_{c,s} = \begin{pmatrix} c & s \\ s & -c \end{pmatrix} \, det(S_{c,s}) = -1$
%
%
% Grafiken einfügen
%
%
\subsubsection{Beweis:}
$A$ ist Isometrie.\\
$A\begin{pmatrix} 1 \\ 0 \end{pmatrix} = \begin{pmatrix} c \\ s \end{pmatrix}  \quad 