\section{Euler-Gerade und Feuerbach-Kreis}
%
%
%
\subsubsection{Definition:}
$a,b,c$ sei ein Dreieck
\begin{enumerate}
	\item Seitenhalbierende: Verbindung von $\vec{a}$ mit $\frac{1}{2}(\vec{b}+\vec{c})$
	\item Mittelsenkrechte: senkrechte auf $ab$ durch $\frac{ab}{2}$
	\item Höhe: durch $a$ ist das Lot von $a$ auf $bc$
\end{enumerate}
%
%
%
\subsubsection{Satz:}
\begin{enumerate}
	\item Die Seitenhalbierenden schneiden sich in einem Punkt: "`Schwerpunkt"'.
	\item Die Mittelsenkrechten schneiden sich in einem Punkt: "`Schwerpunkt"'.
	\item Die Höhen schneiden sich in einem Punkt: "`Schwerpunkt"'.
\end{enumerate}
%
%
%
\subsubsection{Beweis:}
\begin{enumerate}
	\item $\vec{c}\,\mathbb{R}(\frac{a+b}{2}-c)\mathop{=}\limits^{\mathbb{R}=\frac{2}
	{3}}\frac{a+b+c}{3}$\\
	Symmetrie $\Rightarrow$ auf allen Seitenhalbierenden.
	\item $2<a-b,x>=<a-b,a+b>=\Vert a\Vert^{2}-\Vert b\Vert^{2}$\\
	$2<b-c,x>=<b-c,b+c>=\Vert b \Vert^{2}-\Vert c \Vert^{2}$\\
	$2<c-a,x>=<c-a,c+a>=\Vert c \Vert^{2}-\Vert a\Vert^{2}$\\
	$0=0$
\end{enumerate}
$\Rightarrow$ Jede der Gleichungen ist Konsequenz der anderen beiden
%
%
%
\subsubsection{Satz (von Euler):}
$\vec{h}=\vec{s}+2(\vec{s}-\vec{m})$
%
%
%
\subsubsection{Beweis:}
z.z. $\vec{h}=3\vec{s}-2\vec{m}=\vec{a}+\vec{b}+\vec{c}-2\vec{m}$
genauer: $\vec{a}+\vec{b}+\vec{c}-2m$ liegt auf jeder Höhe\\
Höhe durch $c: <a-b,x>=<a-b,c>$\\
$<a-b,a+b+c-2\vec{m}>=\mathop{\underbrace{<a-b,a+b>-<a-b,2\vec{m}>}}\limits_{=0}+<a-b,c>$
%
%
%
\subsubsection{Satz (von Feuerbach):}
Das Dreieck $abc$ mit $s,h,m$\\
%
%Grafik einfügen
%
\subsubsection{Definition:}
Feuerbach-Kreis geht durch Seitenmitte\\
habe Mittelpunkt f
\begin{enumerate}
	\item $\vec{f}=\frac{\vec{h}+\vec{m}}{2}$
	\item Radius d.Feuerbachkreises ist die Hälfte des Umkreisradius. 
	\item $\frac{h+a}{2} , \, \frac{b+h}{2}, \, \frac{h+c}{2}$
	\item Fußpunkt der Höhen auch
\end{enumerate}
%
%
%
\subsubsection{Beweis:}
\begin{enumerate}
	\item $K$ ist Umkreis von %Grafik einfügen
	Lege so, dass $\vec{s}=\vec{0}=\frac{1}{s}(\vec{a}+\vec{b}+\vec{c})=-\frac{1}{4} 
	\mathop{\underbrace{<a-b,a+b>}}\limits_{2<a-b,m>=a-b,\frac{m}{s}>}$\\
	$\Rightarrow \mathop{2}\limits_{\mathop{<a-b,f>}\limits^{\rotatebox{90}
	{=}}}<\frac{a+c}{2}-\frac{b+c}{2},f>=<\frac{a+c}{2}-\frac{b+c}{2},\frac{a+b+2c}{2}>$\\
	$\frac{1}{2} (h+m) \mathop{=}\limits^{\text{Euler}}\frac{1}{2}(-
	m)\mathop{=}\limits^{\Rightarrow f = \frac{-m}{2}} f$
	\item $\leadsto \mathop{\cup}\limits^{\_}$
	\item $\leadsto \mathop{\cup}\limits^{\_}$
	\item %Gafik einfügen
\end{enumerate}
Projeziere $\mathcal{E}$ auf die Gerade $ab$. \\
Dann liegt das Bild von $f$ in der Mitte zum Seitenmittelpunkt und Fußpunkt der Höhe\\
$\Rightarrow \Vert f-\frac{a+b}{2}\Vert = r_{\text{F-Kreis}}=\Vert f-h_{c} \Vert$\\
$\Rightarrow h_{c}$ liegt auf dem F-Kreis